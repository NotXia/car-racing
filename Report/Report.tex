\documentclass[11pt]{article}
\usepackage{algorithm2e}
\usepackage[bottom]{footmisc} 
\usepackage[italian]{babel}
\usepackage[document]{ragged2e}
\usepackage{tabularx}
\justifying
\usepackage{amsfonts, amssymb, amsmath}
\usepackage{cancel}
\usepackage{float}
\usepackage{mathtools}
\usepackage[margin=2cm]{geometry}
% \setcounter{secnumdepth}{0}
\usepackage{hyperref}
\hypersetup{
    colorlinks,
    citecolor=black,
    filecolor=black,
    linkcolor=black,
    urlcolor=black
}
\usepackage{array}
\usepackage{makecell}
\usepackage{hyperref}
\usepackage[noabbrev,capitalize,italian]{cleveref}
\newcommand{\fref}[1]{\hyperref[#1]{\cref{#1}}}
\usepackage{makecell}
\usepackage{etoolbox}
\patchcmd{\thebibliography}{\section*{\refname}}{}{}{}
\usepackage{enumitem}

\tolerance=1
\emergencystretch=\maxdimen
\hyphenpenalty=10000
\hbadness=10000

\begin{document}
\begin{titlepage}
    \begin{center}
        \vspace*{1.5cm}
            
        \Huge
        \textbf{Campionato di\\gare automobilistiche}
            
        \vspace{0.3cm}
        \LARGE
        Report\\[0.2em]

        \vspace{1.5cm}
          
        \begin{minipage}[t]{0.47\textwidth}
            \begin{center}
                \parbox{65mm}{\centering\large {\bf Cheikh Ibrahim $\cdot$ Zaid} \\[0.3em] Matricola: \texttt{0000974909} \\[0.3em] \href{mailto:zaid.cheikhibrahim@studio.unibo.it}{\textit{zaid.cheikhibrahim@studio.unibo.it}}} \\[2em]
            \end{center}
		\end{minipage}
		\hfill
		\begin{minipage}[t]{0.47\textwidth}\raggedleft
            \begin{center}
                \parbox{65mm}{\centering\large {\bf Xia $\cdot$ Tian Cheng} \\[0.3em] Matricola: \texttt{0000975129} \\[0.3em] \href{mailto:tiancheng.xia@studio.unibo.it}{\textit{tiancheng.xia@studio.unibo.it}}} \\[2em]
            \end{center}
		\end{minipage}  
            
        \vspace{9cm}
            
        Anno accademico\\
        $2022 - 2023$
            
        \vspace{0.8cm}
            
            
        \Large
        Corso di Basi di dati\\
        Alma Mater Studiorum $\cdot$ Università di Bologna\\
            
    \end{center}
\end{titlepage}
\pagebreak


\tableofcontents
\newpage


\section{Analisi dei requisiti}
\subsection{Requisiti espressi in linguaggio naturale \textbf{RIFRASARE IN FUTURO, E' GIUSTIFICATO IL TESTO?}}
Si vuole realizzare un database per gestire un campionato di gare automobilistiche. \\
È necessario codificare le gare, le piste su cui si svolgono, i dati relativi ai giri, eventuali infrazioni e i dati sui pit stop. \\
Inoltre, si vogliono memorizzare i dati dei piloti che partecipano e i contratti (presenti e passati) che stipulano con le scuderie. Oltre ai dati relativi alle scuderie, è richiesto registrarne le auto e i meccanici. \\
Infine, si vuole tenere traccia dei controlli di regolarità effettuati dai supervisori (della società che organizza il campionato) e dei dati degli sponsor delle gare e delle singole scuderie. \\[1em]

Per le gare si vuole memorizzare il nome, la data di svolgimento, la pista su cui si corre, il numero di giri previsti, i piloti partecipanti e l'eventuale sponsor. \\
Per le piste si vogliono rappresentare il nome, la nazione e la città di collocazione, la lunghezza (in metri), numero di posti a sedere per gli spettatori. \\
Per i giri si vogliono salvare il tempo impiegato (in secondi), il numero del giro, la gara di appartenenza, il pilota che effettua il giro. \\
Per le infrazioni si vogliono gestire i dati riguardanti il nome e la descrizione e vengono assegnate ad un giro di un pilota sottoforma di penalità (in secondi). \\
Per i pit stop si vogliono rappresentare il tempo delle operazioni, il tempo complessivo (tempo di entrata e uscita + tempo delle operazioni), il giro in cui viene il pilota che viene chiamato ai box e i meccanici che effettuano le operazioni. \\
Per i piloti si vogliono memorizzare il nome, cognome, luogo e data di nascita. \\
Per i contratti si vogliono rappresentare il numero identificativo, il pilota ed il suo numero identificativo, la scuderia, la data d'inizio e di fine, l'auto assegnata e il valore di ingaggio. \\
Per le scuderie si vogliono gestire i dati riguardo la ragione sociale, la nazione della sede principale, l'anno di fondazione, il colore caratterizzante e i vari sponsor. \\ 
Per le auto si vogliono salvare la potenza (in cavalli), velocità massima raggiungibile, la scuderia di appartenenza. \\
Per i meccanici si vogliono memorizzare il nome, cognome, luogo, data di nascita, il ruolo e la scuderia di appartenenza. \\
Per i controlli di regolarità si vogliono tracciare i dati riguardo la data e l'ora, l'auto coinvolta, il supervisore e l'esito. \\
Per i supervisori si vogliono memorizzare il nome, cognome, luogo, data di nascita. \\
Per gli sponsor si vogliono salvare la ragione sociale, la tipologia di azienda, il capitale investito e la nazione della sede principale.

\subsection{Glossario dei termini \textbf{RIVEDERE COLLEGAMENTI}}
\begin{tabularx}{\linewidth}{
        |>{\hsize=0.9\hsize}X|% 10% of 4\hsize 
        >{\hsize=1.8\hsize}X|% 30% of 4\hsize
        >{\hsize=0.6\hsize}X|% 30% of 4\hsize 
        >{\hsize=0.7\hsize}X|% 30% of 4\hsize
           % sum=4.0\hsize for 4 columns
    }
    \hline
    \textbf{Termine} & \textbf{Descrizione} & \textbf{Sinonimi} & \textbf{Collegamenti} \\
    \hline
    Società organizzante & Azienda che organizza un campionato & - & \\
    \hline
    Campionato & Numero definito di gare con classifica & - & Gare \\
    \hline
    Gare & Numero definito di giri & Competizione & Piste, giri, piloti, sponsor \\
    \hline
    Giri & Percorrenza intera di una pista & - & Pilota, gara \\
    \hline
    Piste & Località asfaltata idonea al passaggio di veicoli ad elevata velocità & - & \\
    \hline
    Infrazioni & Eventi irregolari accaduti durante una gara & - & Penalità, giro, pilota \\
    \hline
    Penalità & Tempo ulteriore assegnato al tempo totale & - & \\
    \hline
    Auto & Autoveicolo ad elevata velocità & Veicolo & Scuderia \\
    \hline
    Piloti & Persona che guida un veicolo ad elevata velocità & - & \\
    \hline
    Scuderie & Azienda proprietaria di auto & - & Sponsor \\
    \hline
    Meccanici & Impiegati delle scuderie adibite alla manutenzione dell'auto & - & Scuderia \\
    \hline
    Supervisori & Impiegati della società organizzante adibiti ai controlli di regolarità & - & \\
    \hline
    Controlli (di regolarità) & Controlli effettuati dalla società organizzatrice per garantire la regolarità delle auto & Controlli & Supervisore \\
    \hline
    Sponsor & Azienda che investe per apparire in gare e/o in scuderie & - & \\
    \hline
    Pit stop & Fase di un giro dove l'auto sosta in un'apposita area di pista dove i meccanici effettuano operazioni all'auto & - & Giro, pilota, meccanici \\
    \hline
    Contratto & Accordo stipulato tra un pilota e una scuderia per gareggiare in un campionato & - & Pilota, scuderia \\
    \hline
\end{tabularx}

\subsection{Eliminazione delle ambiguità presenti}
\subsection{Strutturazione dei requisiti}
\subsection{Specifica operazioni}
\begin{enumerate}
    \item Inserire una nuova scuderia (in media 1 volta ogni cinque anni)
    \item Inserire una nuova gara (in media 1 volta all'anno)
    \item Inserire il tempo pit stop (~20 volte per gara)
    \item Inserire il tempo di un giro del pilota sulla pista (~1200 volte per gara)
    \item Inserire un nuovo contratto tra pilota e scuderia (poche volte ogni anno)
    \item Visualizzare gli sponsor di una gara (1 volta per gara)
    \item Visualizzare il pilota con il tempo migliora su una data pista (1 volta per gara)
    \item Visualizzare i piloti e la scuderia con cui gareggiano per una data gara (1 volta per gara)
    \item Visualizzare la classifica (finale o temporanea) di una data gara (~60 volte per gara)
\end{enumerate}

\section{Progettazione concettuale}
\subsection{Identificazione delle entità e relazioni}

\subsection{Definizioni delle entità generalizzabili}
\subsubsection{Definizioni delle persone}
Come persone, sono state identificate le entità supervisori, piloti e meccanici. 
\begin{figure}[H]
    \centering
    \includegraphics[width=10cm]{../er/gare_persone.pdf}
\end{figure}

\subsubsection{Definizioni delle aziende}
Come aziende, sono state identificate le entità scuderia e sponsor. 
\begin{figure}[H]
    \centering
    \includegraphics[width=10cm]{../er/gare_aziende.pdf}
\end{figure}

\subsection{Definizioni delle macro-argomenti}
\subsubsection{Definizioni delle scuderie}
Riguardo le scuderie, sono state identificate le entità: scuderia, contratto, auto, controllo. Oltre a pilota, meccanico, supervisore, sponsor.
\begin{figure}[H]
    \centering
    \includegraphics[width=15.5cm]{../er/gare_scuderie.pdf}
\end{figure}

\subsubsection{Definizioni delle competizioni}
Per il concetto di competizione sono state identificate le entità: gara, pista, giro, infrazione, pit stop. Oltre a pilota, meccanico, sponsor.
\begin{figure}[H]
    \centering
    \includegraphics[width=15.5cm]{../er/gare_gara.pdf} % TODO CORREGGERE ENTITA' MECCANICI->MECCANICO !!!!!!!!!!!!!!!!!!!!!!!!!!!!!!!
\end{figure}

\subsection{Schema finale}
\begin{figure}[H]
    \centering
    \includegraphics[width=15.5cm]{../er/gare.pdf}
\end{figure}


\subsection{Dizionario dei dati}

\begin{center}
\makebox[0cm]{
    \begin{tabular}{ |l|p{5.5cm}|p{5cm}|p{4cm}| }
        \hline
        \textbf{Nome entità} & \textbf{Descrizione} & \textbf{Attributi} & \textbf{Identificatore} \\
        
        \hline
        Pilota & 
        Persona che guida un veicolo & 
        \parbox[t]{\linewidth}{Nome (stringa)\\Cognome (stringa)\\Data di nascita (data)\\Luogo di nascita (stringa)} & 
        Codice fiscale (stringa) \\
        
        \hline
        Meccanico &
        Persona che opera su un veicolo & 
        \parbox[t]{\linewidth}{Nome (stringa)\\Cognome (stringa)\\Data di nascita (data)\\Luogo di nascita (stringa)} & 
        Codice fiscale (stringa) \\

        \hline
        Supervisore &
        Persona che effettua dei controlli di regolarità per conto della società organizzante & 
        \parbox[t]{\linewidth}{Nome (stringa)\\Cognome (stringa)\\Data di nascita (data)\\Luogo di nascita (stringa)} & 
        Codice fiscale (stringa) \\

        \hline
        Scuderia &
        Azienda che stipula contratti con piloti e crea auto da corsa & 
        \parbox[t]{\linewidth}{Colore (stringa)\\Nazione (stringa)\\Anno di fondazione (numero)} & 
        Ragione sociale (stringa) \\

        \hline
        Sponsor &
        Azienda che investe in gare e scuderie & 
        \parbox[t]{\linewidth}{Tipologia (stringa)\\Nazione (stringa)} & 
        Ragione sociale (stringa) \\
        
        \hline
        Contratto &
        Documento stipulato tra un pilota e una scuderia & 
        \parbox[t]{\linewidth}{Data inizio (data)\\Data fine (data)\\Numero pilota (numero)} & 
        Numero contratto (stringa) \\

        \hline
        Auto &
        Autovettura ad elevata velocità di fabbricazione di una scuderia guidata da un pilota &
        \parbox[t]{\linewidth}{Potenza (numero)\\Velocità massima (numero)} & 
        Id (stringa) \\

        \hline
        Controllo &
        Verifica della regolarità di un auto effettuata da un supervisore & 
        \parbox[t]{\linewidth}{Esito (Booleano)} & 
        \parbox[t]{\linewidth}{Data e ora (data)\\Id [Auto] } \\

        \hline
        Gara &
        Competizione dove 20 piloti gareggiano su una pista un numero di giri prestabilito & 
        \parbox[t]{\linewidth}{Data (data)\\Numero giri (numero)} & 
        Nome (stringa) \\

        \hline
        Pista &
        Località asfaltata adatta a ospitare gare ad alta velocità & 
        \parbox[t]{\linewidth}{Nazione (stringa)\\Città (stringa)\\Lunghezza (numero)\\Numero posti (numero)} & 
        Nome (stringa) \\

        \hline
        Giro &
        Singola percorrenza completa di pista & 
        \parbox[t]{\linewidth}{Tempo (numero)} & 
        \parbox[t]{\linewidth}{Numero (numero)\\Nome [Gara]\\Codice fiscale [Pilota]} \\ 

        \hline
        Infrazione &
        Evento irregolare durante una gara & 
        \parbox[t]{\linewidth}{Descrizione (stringa)} & 
        Nome (stringa) \\

        \hline
        Pit stop &
        Fase di gara dove l'auto sosta in una specifica area di pista per permettere ai meccanici di effettuare piccole modifiche & 
        \parbox[t]{\linewidth}{Tempo operazioni (numero)\\Tempo totale (numero)} & 
        Chiavi di [Giro] \\

        \hline
    \end{tabular}
}
\end{center}

\begin{center}
    \makebox[0cm]{
        \begin{tabular}{ |l|p{5.5cm}|p{5cm}|p{4cm}| }
            \hline
            \textbf{Nome relazione} & \textbf{Descrizione} & \textbf{Entità coinvolte} & \textbf{Attributi} \\
            
            \hline
            Svolge & 
            Associa la pista su cui si svolge una gara & 
            \parbox[t]{\linewidth}{Pista (0, N)\\Gara (1, 1)} & 
            \parbox[t]{\linewidth}{-} \\

            \hline
            Promuove & 
             & 
            \parbox[t]{\linewidth}{Gara (0, 1)\\Sponsor (0, N)} & 
            \parbox[t]{\linewidth}{-} \\

            \hline
            Registra & 
             & 
            \parbox[t]{\linewidth}{Gara (1, N)\\Giro (1, 1)} & 
            \parbox[t]{\linewidth}{-} \\

            \hline
            Partecipa & 
             & 
            \parbox[t]{\linewidth}{Gara (20, 20)\\Pilota (0, N)} & 
            \parbox[t]{\linewidth}{-} \\

            \hline
            Investe & 
             & 
            \parbox[t]{\linewidth}{Sponsor (0, N)\\Scuderia (0, N)} & 
            \parbox[t]{\linewidth}{-} \\

            \hline
            Chiamata & 
             & 
            \parbox[t]{\linewidth}{Pit stop (1, 1)\\Giro (0, 1)} & 
            \parbox[t]{\linewidth}{-} \\

            \hline
            Effettua & 
             & 
            \parbox[t]{\linewidth}{Giro (1, 1)\\Pilota (0, N)} & 
            \parbox[t]{\linewidth}{-} \\

            \hline
            Penalizza & 
             & 
            \parbox[t]{\linewidth}{Giro (0, N)\\Penalità (0, N)} & 
            \parbox[t]{\linewidth}{Penalità (numero)} \\

            \hline
            Opera & 
             & 
            \parbox[t]{\linewidth}{Pit stop (1, N)\\Meccanico (0, N)} & 
            \parbox[t]{\linewidth}{-} \\

            \hline
            Stipula & 
             & 
            \parbox[t]{\linewidth}{Pilota (0, N)\\Contratto (1, 1)} & 
            \parbox[t]{\linewidth}{-} \\

            \hline
            Assegna & 
             & 
            \parbox[t]{\linewidth}{Contratto (1, 1)\\Auto (0, N)} & 
            \parbox[t]{\linewidth}{-} \\

            \hline
            Emette & 
             & 
            \parbox[t]{\linewidth}{Contratto (1, 1)\\Scuderia (0, N)} & 
            \parbox[t]{\linewidth}{-} \\

            \hline
            Lavora & 
             & 
            \parbox[t]{\linewidth}{Meccanico (1, 1)\\Scuderia (1, N)} & 
            \parbox[t]{\linewidth}{-} \\

            \hline
            Accerta & 
             & 
            \parbox[t]{\linewidth}{Auto (1, N)\\Controllo (1, 1)} & 
            \parbox[t]{\linewidth}{-} \\
            
            \hline
            Verifica & 
             & 
            \parbox[t]{\linewidth}{Controllo (1,1)\\Supervisore (0, N)} & 
            \parbox[t]{\linewidth}{-} \\
            
            \hline
        \end{tabular}
    }
\end{center}

\subsection{Regole aziendali}
\subsubsection{Regole di vincolo}
\begin{enumerate}[label={RV \arabic*}, leftmargin=4em]
    \item first
    \item second
\end{enumerate}

\subsubsection{Regole di derivazione}
\begin{enumerate}[label={RD \arabic*}, leftmargin=4em]
    \item first
    \item second
\end{enumerate}

\section{Progettazione logica}
\subsection{Tavole dei volumi}
\begin{center}
    \begin{tabular}{ |l|l|l| }
        \hline
        \textbf{Concetto} & \textbf{Tipo} & \textbf{Volume} \\
        
        \hline
        Pilota & Entità & 30 \\
        \hline
        Meccanico & Entità & 150 \\
        \hline
        Supervisore & Entità & 15 \\
        \hline
        Scuderia & Entità & 10 \\
        \hline
        Sponsor & Entità & 50 \\
        \hline
        Contratto & Entità & 1400 \\
        \hline
        Auto & Entità & 20 \\
        \hline
        Controllo & Entità & 55000 \\ % (73 anni di F1 * 20 piloti per gara * 2 (un controllo a inizio gara e uno alla fine) * 20 gare in media per stagione)
        \hline
        Gara & Entità & 1100 \\
        \hline
        Pista & Entità & 50 \\
        \hline
        Giro & Entità & 70000 \\ % (73 anni di F1 * 20 gare in media per stagione * 50 giri in media per gara)
        \hline
        Infrazione & Entità & 20 \\
        \hline
        Pit stop & Entità & 20000 \\
        \hline
    \end{tabular}
        \quad
    \begin{tabular}{ |l|l|l| }
        \hline
        \textbf{Concetto} & \textbf{Tipo} & \textbf{Volume} \\

        \hline
        Svolge & Relazione & 1100 \\
        \hline
        Promuove & Relazione & 700 \\
        \hline
        Registra & Relazione & 70000 \\
        \hline
        Partecipa & Relazione & 22000 \\
        \hline
        Investe & Relazione & 300 \\
        \hline
        Chiamata & Relazione & 20000 \\
        \hline
        Effettua & Relazione & 70000 \\
        \hline
        Penalizza & Relazione & 8000 \\
        \hline
        Opera & Relazione & 300000 \\ % (volume meccanici 15 * volume pit stop 20000)
        \hline
        Stipula & Relazione & 1400 \\
        \hline
        Assegna & Relazione & 1400 \\
        \hline
        Emette & Relazione & 1400 \\
        \hline
        Lavora & Relazione & 150 \\
        \hline
        Accerta & Relazione & 55000 \\
        \hline
        Verifica & Relazione & 55000 \\
        \hline
    \end{tabular}
\end{center}
\subsection{Tavola delle operazioni}
\begin{center}
    \begin{tabular}{ |l|l| }
        \hline
        \textbf{Operazione} & \textbf{Frequenza} \\

        \hline
        1 & 1 volta all'anno \\
        \hline
        2 & 1 volta all'anno \\
        \hline
        3 & 1 volta all'anno \\
        \hline
        4 & 1 volta all'anno \\
        \hline
        5 & 1 volta all'anno \\
        \hline
        6 & 1 volta all'anno \\
        \hline
        7 & 1 volta all'anno \\
        \hline
        8 & 1 volta all'anno \\
        \hline
        9 & 1 volta all'anno \\
        \hline
        10 & 1 volta all'anno \\
        \hline
        11 & 1 volta all'anno \\
        \hline
        12 & 1 volta all'anno \\
        \hline
        13 & 1 volta all'anno \\
        \hline
        14 & 1 volta all'anno \\
        \hline
        15 & 1 volta all'anno \\
        \hline
    \end{tabular}
\end{center}


\subsection{Ristrutturazione dello schema concettuale}
\subsection{Normalizzazione}
\subsection{Traduzione verso il modello relazionale}

\section{Codifica SQL}
\subsection{Definizione dello schema}
\subsection{Codifica delle operazioni}

\section{Testing}

\end{document}
